\documentclass[a4paper, 12pt]{article}
\usepackage{color}
\usepackage{threeparttable}

\begin{document}

\title{\LaTeX\ How To}
\author{Matthew W. Brockley}
\date{\today}
\maketitle

\pagenumbering{roman}
\tableofcontents
\newpage
\pagenumbering{arabic}

\section{Introduction}
This is the introduction.

\section{Document Structure Example}

\subsection{Stage 1}
\label{sec1}
demo 1
\subsection{results}
This is how you make a reference to a label that you created.  Using the code from \ref{sec1} on page \pageref{sec1}.

\section{Typesetting Text}

\subsection{Font Effects}
\textit{text in italics} \\
\textsl{slanted text} \\
\textsc{smallcaps} \\
\textbf{words in bold} \\
\texttt{words in teletype} \\
\textsf{sans serif} \\
\textrm{roman words} \\
\underline{underlines words} \\

\subsection{Colored Text}
Colored text can be used by using the \\ \texttt{\textbackslash usepackage\{color\}} \\ package.

This is an example or writing the word {\color{red}fire} in the color red.
The following colors are recognized by the package: {\color{red}red}, {\color{blue}blue}, {\color{cyan}cyan}, {\color{yellow}yellow}, {\color{green}green}, {\color{magenta}magenta}, black, white. \\

More information on color is in the \LaTeX\ workbook.

\subsection{Font Sizes}
Font Size Examples:

{\tiny tiny words}
{\scriptsize scriptsize words}
{\footnotesize footnote words}
{\normalsize  normal words}
{\large large words}
{\Large Large words}
{\LARGE LARGE words}
{\huge huge words}




\subsection{Lists}

Example of a list: \\

\begin{enumerate}
\item First thing
\item Second thing
\begin{itemize}
\item A sub-thing
\item Another sub-thing
\end{itemize}
\item Third thing
\end{enumerate}

The bullet format can be changed by using \texttt{\textbackslash item[x]} where x is the bullet character.

\subsection{Comments \& Spacing}
 % example of a comment 
There is a comment in this line, %effect subsides when enter is pressed
but not in this one.

\subsection{Special Characters}

Item \#1A\textbackslash642 costs \$8 \& is sold at a \~{}10\% profit.

\section{Tables}

This is a demonstration of how to make some tables: \\

\begin{tabular}{l|r|r}
Item & Quantity & Price (\$) \\
\hline
Nails & 500 & 0.34 \\
Wooden boards & 100 & 4.00 \\
Bricks & 240 & 11.50 \\
\end{tabular}

\begin{tabular}{|l|c c c|}
\hline
{} & {} & Year & {} \\
\cline{2-4}
City & 2006 & 2007 & 2008 \\
\hline
London & 45789 & 46551 & 51298 \\
Berlin & 34549 & 32543 & 29870 \\
Paris & 49835 & 51009 & 51970 \\
\hline
\end{tabular}

% Table created by stargazer v.5.2.2 by Marek Hlavac, Harvard University. E-mail: hlavac at fas.harvard.edu
% Date and time: Wed, Dec 09, 2020 - 9:58:34 PM
\begin{table}[!htbp] \centering 
  \caption{} 
  \label{} 
\begin{tabular}{@{\extracolsep{5pt}}lccccccc} 
\\[-1.8ex]\hline 
\hline \\[-1.8ex] 
Statistic & \multicolumn{1}{c}{N} & \multicolumn{1}{c}{Mean} & \multicolumn{1}{c}{St. Dev.} & \multicolumn{1}{c}{Min} & \multicolumn{1}{c}{Pctl(25)} & \multicolumn{1}{c}{Pctl(75)} & \multicolumn{1}{c}{Max} \\ 
\hline \\[-1.8ex] 
ï..NADPH\_conc & 12 & 0.002 & 0.003 & 0 & 0.000 & 0.001 & 0 \\ 
rate\_min & 12 & 79.934 & 42.827 & 0 & 58.0 & 104.6 & 133 \\ 
rate\_sec & 12 & 1.332 & 0.714 & 0 & 1.0 & 1.7 & 2 \\ 
kcat & 12 & 0.181 & 0.097 & 0 & 0.1 & 0.2 & 0 \\ 
\hline \\[-1.8ex] 
\end{tabular} 
\end{table}

% Table created by stargazer v.5.2.2 by Marek Hlavac, Harvard University. E-mail: hlavac at fas.harvard.edu
% Date and time: Wed, Dec 09, 2020 - 10:04:21 PM
\begin{table}[!htbp] \centering 
  \caption{} 
  \label{} 
\begin{tabular}{@{\extracolsep{5pt}}lccccccc} 
\\[-1.8ex]\hline 
\hline \\[-1.8ex] 
Statistic & \multicolumn{1}{c}{N} & \multicolumn{1}{c}{Mean} & \multicolumn{1}{c}{St. Dev.} & \multicolumn{1}{c}{Min} & \multicolumn{1}{c}{Pctl(25)} & \multicolumn{1}{c}{Pctl(75)} & \multicolumn{1}{c}{Max} \\ 
\hline \\[-1.8ex] 
rating & 30 & 64.633 & 12.173 & 40 & 58.8 & 71.8 & 85 \\ 
complaints & 30 & 66.600 & 13.315 & 37 & 58.5 & 77 & 90 \\ 
privileges & 30 & 53.133 & 12.235 & 30 & 45 & 62.5 & 83 \\ 
learning & 30 & 56.367 & 11.737 & 34 & 47 & 66.8 & 75 \\ 
raises & 30 & 64.633 & 10.397 & 43 & 58.2 & 71 & 88 \\ 
critical & 30 & 74.767 & 9.895 & 49 & 69.2 & 80 & 92 \\ 
advance & 30 & 42.933 & 10.289 & 25 & 35 & 47.8 & 72 \\ 
\hline \\[-1.8ex] 
\end{tabular} 
\end{table}

\begin{table}[!htbp] \centering 
  \caption{Oxygraph Table of Data} 
  \label{} 
\begin{tabular}{@{\extracolsep{5pt}} ccccccc} 
\\[-1.8ex]\hline 
\hline \\[-1.8ex] 
ï..NADPH\_conc & rate\_min & rate\_sec & kcat & X & X.1 & X.2 \\ 
\hline \\[-1.8ex] 
$0$ & $0$ & $0$ & $0$ &  &  &  \\ 
$0.00001$ & $22.700$ & $0.378$ & $0.051$ &  &  &  \\ 
$0.00002$ & $27.090$ & $0.452$ & $0.061$ &  &  &  \\ 
$0.00004$ & $68.320$ & $1.139$ & $0.155$ &  &  &  \\ 
$0.0001$ & $78.150$ & $1.302$ & $0.177$ &  &  &  \\ 
$0.0001$ & $89.500$ & $1.492$ & $0.202$ &  &  &  \\ 
$0.0003$ & $97.190$ & $1.620$ & $0.220$ &  &  &  \\ 
$0.001$ & $99.260$ & $1.654$ & $0.224$ &  &  &  \\ 
$0.001$ & $101.800$ & $1.697$ & $0.230$ &  &  &  \\ 
$0.002$ & $112.900$ & $1.882$ & $0.255$ &  &  &  \\ 
$0.005$ & $129.400$ & $2.157$ & $0.293$ &  &  &  \\ 
$0.009$ & $132.900$ & $2.215$ & $0.301$ &  &  &  \\ 
\hline \\[-1.8ex] 
\end{tabular}

\begin{tablenotes}

\item \footnotesize{Collected Data using an Hanstech oxygraph probe to determine inital rate values for a variety of NADPH concentrations.}

\end{tablenotes}
 
\end{table} 

\section{Figures}

\section{Equations}
\subsection{Inserting Equations}

Math mode is accessed bracketing text with \$s.  Use two \$s to get the equation onto its own line.

$$1+2=3$$

Labeled Equation:
\begin{equation}
1+2=3
\end{equation}

Equation Array:
\begin{eqnarray}
a & = & b + c\\
{} & = & y - z
\end{eqnarray}

\subsection{Mathematical Symbols}
Superscript and Subscript:\\
$n^2$\\
$n_2$\\
$b_{a-2}$\\

Fractions:
$\frac{numerator}{denominator}$\\
Fractions can be nested\\

Roots:
$\sqrt[index]{expression}$\\

Summations and Integrals:
$$\sum_{x=1}^5 2^x$$\\
$$\int_a^b f(x)dx$$

\subsection{Examples}
\begin{equation}
e = mc^2
\end{equation}
\begin{equation}
\pi = \frac{c}{d}
\end{equation}
\begin{equation}
\frac{d}{dx}e^x = e^x
\end{equation}
\begin{equation}
f(x) = \sum_i = 0^\infty \frac{f^{(i)} (0)}{i!}x^i
\end{equation}
\begin{equation}
x = \sqrt{\frac{x_i}{z} y}
\end{equation}

\section{Inserting References}



\end{document}